\documentclass[hyperref={unicode=true}]{beamer}
\usepackage{multirow}
\usepackage{amsmath} % used for boldsymbol.
\renewcommand{\vec}[1]{\boldsymbol{#1}} % Uncomment for BOLD vectors.
\usepackage[slantfont,boldfont]{xeCJK}
\setCJKmainfont{SimSun}
\setCJKmonofont{FangSong}
\setCJKsansfont{SimHei}
\usetheme{Darmstadt}
\usecolortheme{beaver}
\usepackage{booktabs}
\usepackage{tikz}
\setbeamertemplate{theorems}[numbered]
\newtheorem{mytht}{\bf 定理}
\theoremstyle{definition}
\newtheorem{mydef}[]{\bf 定义}
\theoremstyle{proof}
\newtheorem{myprf}[]{\bf 证明}
\usepackage{ulem}
\usepackage[T1]{fontenc}
\usepackage{inconsolata}
\usepackage{subfigure}
\makeatletter
% there's no italic/slanted for Inconsolata
\@namedef{T1/zi4/m/it}{<->ssub*zi4/m/n}
\@namedef{T1/zi4/b/it}{<->ssub*zi4/b/n}
\@namedef{T1/zi4/bx/it}{<->ssub*zi4/b/n}
\@namedef{T1/zi4/m/sl}{<->ssub*zi4/m/n}
\@namedef{T1/zi4/b/sl}{<->ssub*zi4/b/n}
\@namedef{T1/zi4/bx/sl}{<->ssub*zi4/b/n}
\makeatother

\graphicspath{{figures/}}
\input zhwinfonts
\begin{document}
\setbeamertemplate{caption}[numbered]
\renewcommand\figurename{图}
\renewcommand\tablename{表}
\renewcommand\contentsname{\centering 目录}


%%------------------------------------------
\title{基于深度学习的驾驶场景语义分割算法的研究}
\subtitle{ParaBN}
\author{马玉坤}
\institute{哈尔滨工业大学计算机科学与技术学院}
\date{2019年3月28日}
%%------------------------------------------

\begin{frame}\titlepage\end{frame}

  \begin{frame}\tableofcontents\end{frame}

    \begin{frame}{语义分割上的效果}
      \begin{figure}[!h]
        \centering
        \includegraphics[width=4in]{parabn/fcn8s.png}
        \caption{用ParaBN使用Cityscapes数据与GTA数据一起训练}
        \label{fig:parabn_fcn8s}
      \end{figure}
    \end{frame}

    \begin{frame}{图像分类上的效果}
      \begin{figure}[!h]
        \centering
        \includegraphics[width=\textwidth]{parabn/class.png}
        \caption{用ParaBN进行分类任务}
        \label{fig:parabn_class}
      \end{figure}
    \end{frame}

\section{数据集}


\begin{frame}\frametitle{数据集}
  \begin{block}{Cityscapes}
    \begin{figure}[!h]
      \setlength{\subfigcapskip}{-1bp}
      \centering
      \begin{minipage}{\textwidth}
        \centering
        \subfigure[Cityscapes图像1]{\includegraphics[width=0.4\textwidth]{cs_1.png}}
        \hspace{2em}
        \subfigure[Cityscapes标签1]{\includegraphics[width=0.4\textwidth]{cs_label1.png}}
      \end{minipage}
      \begin{minipage}{\textwidth}
        \centering
        \subfigure[Cityscapes图像2]{\includegraphics[width=0.4\textwidth]{cs_2.png}}
        \hspace{2em}
        \subfigure[Cityscapes标签2]{\includegraphics[width=0.4\textwidth]{cs_label2.png}}
      \end{minipage}
      \caption{Cityscapes结果}\label{fig:cs_samples}
    \end{figure}
  \end{block}
\end{frame}

\begin{frame}\frametitle{数据集}
  \begin{block}{GTAV}
    \begin{figure}[!h]
      \setlength{\subfigcapskip}{-1bp}
      \centering
      \begin{minipage}{\textwidth}
        \centering
        \subfigure[GTA图像1]{\includegraphics[width=0.4\textwidth]{gta_1.png}\label{fig:gta1}}
        \hspace{2em}
        \subfigure[GTA标签1]{\includegraphics[width=0.4\textwidth]{gta_label1.png}\label{fig:gta1l}}
      \end{minipage}
      \begin{minipage}{\textwidth}
        \centering
        \subfigure[GTA图像2]{\includegraphics[width=0.4\textwidth]{gta_2.png}\label{fig:gta2}}
        \hspace{2em}
        \subfigure[GTA标签2]{\includegraphics[width=0.4\textwidth]{gta_label2.png}\label{fig:gta2l}}
      \end{minipage}
      \caption{GTA结果}\label{fig:gta}
    \end{figure}
  \end{block}
\end{frame}


\begin{frame}\frametitle{数据集}
  \begin{block}{ImageNet}
    \begin{figure}[!h]
      \setlength{\subfigcapskip}{-1bp}
      \centering
      \begin{minipage}{\textwidth}
        \centering
        \subfigure[熊猫图像1]{\includegraphics[width=0.2\textwidth]{imagenet/panda1.png}}
        \subfigure[熊猫图像2]{\includegraphics[width=0.2\textwidth]{imagenet/panda2.png}}
        \subfigure[熊猫图像3]{\includegraphics[width=0.2\textwidth]{imagenet/panda3.png}}
      \end{minipage}
      \begin{minipage}{\textwidth}
        \centering
        \subfigure[花图像1]{\includegraphics[width=0.2\textwidth]{imagenet/flower1.png}}
        \subfigure[花图像2]{\includegraphics[width=0.2\textwidth]{imagenet/flower2.png}}
        \subfigure[花图像3]{\includegraphics[width=0.2\textwidth]{imagenet/flower3.png}}
      \end{minipage}
      \caption{Imagenet图片示例}\label{fig:imagenet_samples}
    \end{figure}
  \end{block}
\end{frame}


\begin{frame}\frametitle{数据集}
  \begin{block}{CIFAR-100}
    \begin{figure}[!h]
      \centering
      \includegraphics[width=3in]{cifar100.png}
      \caption{CIFAR-100数据示例}
      \label{fig:cifar100_samples}
    \end{figure}
  \end{block}
\end{frame}

\section{语义分割}

\begin{frame}\frametitle{FCN8s网络}
  \begin{figure}[!h]
    \centering
    \includegraphics[width=0.8\textwidth]{fcn8s_arch.jpeg}
    \caption{FCN8s网络结构}
    \label{fig:fcn8s_arch}
  \end{figure}
  将不同池化层的结果进行上采样,然后结合这些结果来优化输出。\\
  backbone: VGG16
\end{frame}


\begin{frame}\frametitle{DeepLab v2网络}
  \begin{block}{DeepLab v2}
    \begin{figure}[!h]
      \centering
      \includegraphics[width=0.5\textwidth]{segmentation/deeplabv2_curve.png}
      \caption{DeepLabv2训练曲线}
      \label{fig:deeplabv2_curve}
    \end{figure}
  \end{block}
\end{frame}

\section{ParaBN}

\begin{frame}{什么是ParaBN}
  \begin{itemize}
  \item 将本来网络中的一个BatchNorm层,变成若干个BatchNorm层的组合
  \item 对不同组的输入,选择不同的BatchNorm层进行forward和backward
  \end{itemize}
\end{frame}

\begin{frame}{ParaBN与语义分割}
  \begin{figure}[!h]
    \centering
    \includegraphics[width=\textwidth]{parabn/parabn_csgta.png}
    \caption{使用Cityscapes数据与GTA数据一起训练}
    \label{fig:parabn_csgta}
  \end{figure}
\end{frame}

\begin{frame}{语义分割效果}
  \begin{figure}[!h]
    \centering
    \includegraphics[width=4in]{parabn/fcn8s.png}
    \caption{训练结果}
    \label{fig:parabn_fcn8s}
  \end{figure}
\end{frame}


\begin{frame}{语义分割效果}
  \begin{figure}[!h]
    \centering
    \includegraphics[width=2.5in]{parabn/fcn8s_curve.png}
    \caption{训练曲线}
    \label{fig:parabn_fcn8s}
  \end{figure}
\end{frame}

\begin{frame}{ParaBN与图像分类}
  \begin{figure}[!h]
    \centering
    \includegraphics[width=\textwidth]{parabn/parabn_size.png}
    \caption{用ParaBN进行多尺度图片分类任务}
    \label{fig:parabn_size}
  \end{figure}
\end{frame}

\begin{frame}{图像分类结果}
  \begin{figure}[!h]
    \centering
    \includegraphics[width=\textwidth]{parabn/class.png}
    \caption{用ParaBN进行分类任务}
    \label{fig:parabn_class}
  \end{figure}
\end{frame}

\section{数据扩充}

\begin{frame}{动机}
  \begin{itemize}
  \item 图片白天多,晚上少。晴天多,雨天少。
  \item 白天->晚上,晴天->雨天
  \item 扩充数据集
  \end{itemize}
\end{frame}

\begin{frame}{AdaIN}
  \begin{figure}[!h]
    \centering
    \includegraphics[width=\textwidth]{adain_arch.jpg}
    \caption{AdaIN网络结构}
    \label{fig:adain_arch}
  \end{figure}
\end{frame}


\begin{frame}{晴天转雨雪天}
  \begin{figure}[htbp]
    \subfigure[雨天]{\includegraphics[width=0.3\textwidth]{style_trans/sun2rain_2.jpg}}
    \subfigure[雪天]{\includegraphics[width=0.3\textwidth]{style_trans/sun2rain_1.jpg}}
    \subfigure[晚上]{\includegraphics[width=0.3\textwidth]{style_trans/sun2rain_3.jpg}}
    \caption{从“晴天”风格转换成“雨雪天”风格}
    \label{fig:sun2rain}
  \end{figure}
\end{frame}

\begin{frame}{效果}
  \begin{figure}[!h]
    \centering
    \includegraphics[width=\textwidth]{style_trans/day2night_res.png}
    \caption{加入转换后的夜晚数据后}
    \label{fig:day2night_res}
  \end{figure}
\end{frame}

\begin{frame}{效果}
  \begin{figure}[!h]
    \centering
    \includegraphics[width=\textwidth]{style_trans/sun2rain_res.png}
    \caption{加入转换后的雨天数据后}
    \label{fig:sun2rain_res}
  \end{figure}
\end{frame}

\begin{frame}\frametitle{}
  \vspace{0.2in}
  \centering{\Huge 谢谢}
\end{frame}

%% \begin{frame}[allowframebreaks]
%%         \frametitle{参考文献}
%%         \bibliographystyle{unsrt}
%%         \bibliography{ref.bib}
%% \end{frame}

\end{document}
